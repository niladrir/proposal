%%% =================chapter 5 starts here ===========================
%%

\chapter{Application to Data Study} \label{ch:application} The department is getting a huge dataset from General Electric, better known as GE. The data is supposed to contain large medical databases to inform the development and validation of heart failure and cardiovascular risk engines. The goal is to develop exploratory, visualization and statistical models for the analysis and assessment of data collected over multiple cardiovascular studies. 

My plan is to apply some of the methods we proposed here for drawing inference while doing modeling. Since visual inference require less technical knowledge to make a decision, I believe that would be an elegant way of presenting my result before the management. The goal is to see how visual inference methods work for this real data set that Novartis is now working with. 

\section{About Data} One of the challenge is to clean the data as the data set is huge. The goal is to extract useful information from this data set which could be used for further modeling. In this section we will describe the data set and present the anomalies if we notice any as well as provide suggestions how to make use of this data.

\section{Exploratory Study} This section will present some exploratory analysis about this huge data set. The plan is to come up with some validation criteria that would guarantee the usefulness of the data for modeling purpose. Visual inference technique will be applied for this validation.

\section{Results obtained from Visual Inference} This section will present results obtained from applying visual inference technique to the data.%%
%%% ===============chapter 6 starts here ============================
